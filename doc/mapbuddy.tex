% -*-latex-*-
% Typical article template

\documentclass[a4paper,10pt]{article}
\usepackage{amsmath}
\usepackage{amstext}
\usepackage{amsbsy}
\title{ Map Buddy: A Mobile Map Application}

\author{Eray \"Ozkural and Kartal Tabak}

\date{\today}

\begin{document}

\maketitle

\section{Introduction}

Evolution of mobile technology has accelerated a trend to access geographic
knowledge using small mobile units. With the advanced processing power,
wireless communication and persistent storage in mobile units, it becomes
desirable to take advantage of locational knowledge. Although there are
several projects on this subject, few services have been designed with the
mobile users in mind. Map Buddy is a mobile application for access to
real-time city maps and querying routes for travel. We aim to fulfill a
specific map service to this end.

Map Buddy will be a client/server application with all expensive processing
and storage offloaded to a server. The clients will be mobile units which
are naturally equipped with limited display capability. Mobile users will be
able to make map queries and graphically browse their physical neighborhood.
They will also be able to query routes and retrieve the result in graphical
and textual form. Also considered is a service to let users update tokens on
the map for a kind of geographical whiteboard among users sharing the
server. For instance, road users may indicate where an accident has occured.
For such use, it may also be desirable to allow groups of users since all
users may not trust each other.

We will design a simple application protocol to define these services. The
server will be implemented in an efficient language for a daemon that can
run over a UNIX machine. The first mobile client will be implemented in Java
language, presumably for laptops. We are essentially abandoning the idea of
using a huge, complex and costly GIS for the sake of a self-contained, open
and efficient modern code suitable for mobile applications which can be
rapidly deployed at little or no cost. We expect that the demonstration
platform of our project will be eventually applicable to a detailed data of
Ankara city.

\section{Background}

Graph theoretical algorithms have improved to a point where we can
answer all the queries of the system described above. Almost any
important path query algorithm except k-th shortest path algorithms
can be found in an algorithm book for which \cite{algo} sets a decent
standard. K-th shortest path algorithm has been refined a bit over the
previous ten years, and we will use a reasonable algorithm among the
alternatives. One of the most recent and comprehensive works on this
area is \cite{eppstein}.  There are various attempts to provide
geographic information. These studies are generally city based, and as
an object of our study, Ankara has several classes of such
information.  The zoomable map of the municipality of Ankara is the
most detailed one. It can be accessed via \cite{burc}. But this map
does neither concern about mobility, nor provides graph based
information such as "How to go from X to Y?" since the map is
implemented only as layers of images. Another project is by Opel
International. \cite{opel} They supplied a graph based routing service
via web, but this service is not specific to Ankara and only the main
streets are included in the map.  Besides, it is limited to a path
query service. GSM service providers are providing similar services,
queries such as "Where is the nearest hospital?", but they do not
include the routing information.  Yet another project is implemented
by ASKI. \cite{aski} It is very similar to \cite{burc}, with a bit
differences in the user interface and queries. This project has the
same deficiencies. A popular web service for map queries is
maps.yahoo.com which allows personalized views of the maps with
bookmarks on the map and integrates satellite/aerial photos as well as
US road information down to street level. \cite{yahoo} This system has
the amazing feature of querying for an arbitrary postal address
accurately. All of these systems are addressing certain needs and
classes of users, but none of them actually solves the problems and
needs of a mobile user as we have described.


\section{Map conceptualization and structure}

\subsection{Location information for mobile units}

We assume in our work that the mobile platform in question already provides
for a user location service that can work at the resolution of a city-block
or better. It is sufficient that the mobile unit can query for its own
location in our system.

\subsection{Basic information requirements}

Maps contain geographic information with varying objectives, i.e. a road map
as distinguished from a military map. We first need to establish the kinds
of available information for specifying the content of our map. Our map
application framework is designed to constitute a basic service for building
more sophisticated and specific applications targeting mobile users who live
in cities and towns. We can easily identify at least three kinds of entities
that will be of interest to mobile users: people, places and roads.  While a
traditional map service is usually regarded as a discovery tool, it may also
serve as a place for dynamic geographic information. We should support all
such potential uses at least in the design phase in order to avoid
indefinitely blocking future extensions which will most certainly be
progressing toward more mobility and networking.

With these considerations, we can argue that the map should facilitate at
least the following classes of information.

\begin{description}
\item[Point location] A unit of information which has geographic meaning
  when expressed with only a single coordinate. We can organize points
  according to their types. Each point can be given a type, in which case a
  corresponding graphical representation will be used for it. This kind of
  information may be used for bookmarks on the map, location of individuals,
  vehicles and road intersections.
\item[Region location] A region on the map which corresponds to an object in
  the physical world. Again a type should be given to indicate proper
  rendering of the polygon on the display device together with any
  additional attributes such as name or date of creation. For example, a
  city block or a park may be conveniently represented by such a system.
\item[Roads/routes] The motor vehicle roads, railroads, walking lanes are
  indispensable in an urban map. A road contains extra information with
  regards to its directionality, length and type such as highway or
  residential road. \footnote{It is also considerable that entire routes may
    be stored on the map while not a strict necessity.}
\end{description}

From a close inspection of geographic information units, we can see the
following attributes which are possibly common to each kind of entity:
\begin{description}
\item[Type] The type of entity indicates the ways in which the entity
  differs conceptually from other entities in the same class of geographic
  information. That is how one distinguishes a building from a parking lot,
  and the type information is one that is often considered as
  \emph{overlays} of maps. For an analogy, consider that we have transparent
  maps of train stations and gas stations which, when overlayed on the basic
  urban map, visually compose a station map containing both types of
  entities.
\item[Name] Significant entities often have names. For instance, the T\"ubitak
  building is recognized among Ankara citizens because of its visual
  domination and institutional importance.
\item[Color/Icon/Texture] For certain structures, it may be well possible to
  denote specific kinds of 2d graphical representation for it may help
  easier visual identification. For instance, METU buildings may be
  represented by a small METU icon in addition to their polygonal extent.
\end{description}  

Note that this list may be extended quite easily. We are consciously
excluding those features which may require more processing power and network
bandwidth than can be afforded by typical mobile units up to 2005.

In addition, a zoom threshold per each type can be instrumental in
specifying what is to be displayed at a given zoom level. We can use a
simple numerical measure to aid us in resolution management when the user
zooms in and out. Above a specified zoom-level, a type of unit i.e.
residential blocks, may be eliminated from the map to avoid cluttering.

\subsection{Operations}

We must foresee the basic kinds of operations we would like to support in
advance.  These operations should also be efficient since several users may
be executing them at once.

We will perform the following kinds of queries on the map:
\begin{description}
\item[Rectangular query] a portion of the map, possibly subject to detail
  management for display on a mobile device. We would ideally like the user
  to be able to zoom in/out and move in eight directions easily and without
  difficulty.
\item[Route query] a suitable path from point A to point B. A good path is
  typically one that is shortest and is without obstacles.
\end{description}

We would also like to perform the following kinds of updates:
\begin{description}
\item[Update point location] Required for people at least. A major emphasis
  of our project is the dynamic nature of urban maps. Mobile users may tend
  to travel quite frequently in large cities, and such information can be
  useful even in a relatively small scale.
\item[Update region location] Required for modifying information about a new
  club, for instance.
\item[Update route] For people who would like to communicate and store
  routes.
\end{description}

However, for a proof-of-concept implementation we should be content with the
first kind of update operation which easily allows us to implement bookmarks
and real-time individual location display.

\subsection{Data structures and algorithms}

Every physical point must contain coordinates in a uniform way, we can use a
standard co-ordinate system like GPS for this purpose.

To support point and rectangular queries, we should store all the this
information in an efficient way using appropriate geometric structures. Most
of our requirements are satisfied by existing algorithms and data structures
found in a standard computational geometry text such as \cite{compgeom}. The
implementation of most of these algorithms are publicly available through free
software libraries such as \cite{cgal}. 

\begin{description}
\item[points] a set of labeled objects, since some points in our definition
  are mobile (i.e. persons) and some other points are accessed by other
  structures (i.e. street intersections). To carry out the map queries, we
  require a 2d point set data structure which allows various geometric
  queries including isorectangular queries and k-nearest neighbors in an
  efficient fashion. The design of such a combinatorial software platform
  may be found in \cite{leda}. The attributes of name and type are stored
  per point and an auxiliary index at least over name attributes must be
  supported. \footnote{For instance, we would like to query a point with its
  name. We may also consider in the future such queries as ``Where is the
  nearest person?``}
\item[regions] a set of polygons forming a planar graph with efficient
  storage.  This can be implemented with a standard DCEL \footnote{A common
    representation for planar graphs}. The points of regions in the planar
  graph correspond to the 2d point set structure. Therefore, once we obtain
  a rectangular query over points, we can extract from the DCEL the subset
  of the planar graph whose polygons should be subsequently clipped for
  preparing display on the mobile unit.
\item[routes] transportation information can be represented with a weighted
  directed graph whose vertices are road intersections/points \footnote{like
    intersection of 1st Ave. and 20th St.} and edges are road segments
  \footnote{which has an integer index in the planar graph structure}.  The
  weight of an edge represents the length of a road segment. The directions
  of edges correspond to road directions naturally.  indicated previously,
  road intersections and end points are represented by our 2d point set and
  the road segments are represented as regions in DCEL.
\end{description}

Let us now describe how we can calculate the route from current location to
a destination. First, the input coordinates must be converted into vertices
of the transportation graph. For this discretization, we can use a k-nearest
points query on the point set. Consequently, we can compute either a
shortest time path or a shortest length path over the transportation graph.
We will use k-shortest paths algorithm on a time-weighted or length-weighted
graph for this purpose. The time elapsed over an edge of the time-weighted
graph can be taken to be an average speed determined according to road type
and segment shape multiplied by road length. \footnote{We may simplify this
  calculation by ignoring segment shape or ramps and simply taking the speed
  to be constant for a road type} The resulting paths can then be
conveyed either graphically or textually.

In addition to this, in a future system we may imagine that the obstacles
are taken into account in the calculation of paths. We suggest a weight
adjustment scheme for increasing the weight (time or length) of edges with a
suitable factor temporarily, where traffic jams or accidents are reported.

\subsection{Map views}

A group is a set of users. Any user can start a group. To join a group, an
existing member must approve the join request. To remove a user, a user must
take action and another must confirm. A group which has not been logged on
to by any member for a long period of time \footnote{For instance a year} is
automatically deleted from the system. The group operations should be very
simple so as not to betray the efficiency of mobile access.

Each group effectively views a different map. This is achieved by augmenting
a common map with group-level information. For instance, a large group of
friends can use this to mark relevant events for their interests. Likewise,
for personal maps (such as bookmarks).

Each user can make his own geographical information visible or invisible.
This can also be done on a per-group basis.

The demonstration platform will have a single group to which all users are
automatically added to. However, it should be an easy matter to provide for
the full design suggested in this section.

\subsection{User Interface}

The UI design of the client is fairly obvious, and we plan to take the
interface of \cite{yahoo} as a model to base ours upon. On startup, the
system centers a street-level map on current user location. User can move
about the map either by choosing a new center or one of the navigation
buttons for eight main directions. The user can also change the zoom level
from a predetermined list of appropriate zoom levels, or zoom in and out of
the current level. The user should be able to query points by text
\footnote{However, we will not implement a fault-tolerant scheme for text
  queries for demonstration platform}. The bookmarks can be managed in
typical web browser fashion. In route queries, we let the user select
the source and destination directly, from bookmarks or current location.
The user can also change visibility status.

\section{Architecture and Platform}

The architecture we have chosen is a simplistic client/server
separation in which the RPCs are sent over TCP using an application
protocol. The target platform for the server is considered to be an
ordinary UNIX server which will run a daemon to process incoming
requests. The requests may be at a rate of several queries per second.
Therefore, we require a language that can be used to implement
efficient algorithms.  C++ is an efficient imperative language and the
free software g++ compiler is well supported for a myriad of platforms
which provides a very large subset of ISO C++ standard. Of course, the
most important factor for choosing C++ for server process is the
availability of free software libraries of computational geometry code
such as CGAL.  \cite{cgal}

While Java 2 may be thought of as a suitable platform for the
graphical client in the upcoming year 2004, it may not be an optimal
choice for a server which will support hundreds, if not thousands of
threads per server.  For the GUI, however, we project that a Java 2
solution will be available for a good number of handheld devices by
the time deliverables will be ready.

\bibliography{mapbuddy} \bibliographystyle{plain}

\end{document}
